\section{Introduction}

\par Traditional methods of assessing performance in hockey primarily rely on raw statistics, which often fail to account for the contextual factors influencing scoring opportunities.
These conventional metrics, such as goals, assists, or shots on goal, provide limited insights into the true nature of a team's offensive and defensive capabilities. For example, a player’s goal count may not accurately reflect their contribution to scoring chances, nor does it capture the quality of those chances. Similarly, shot attempts do not distinguish between high-quality scoring opportunities and low-percentage shots.
To address these limitations, the concept of Expected Goals (xG) has emerged as a more sophisticated metric for evaluating scoring potential.
